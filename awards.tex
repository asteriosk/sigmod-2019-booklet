%!TEX root = booklet.tex

\newcommand{\personphoto}[1]{{\includegraphics[width=21mm,height=25mm,keepaspectratio]{#1}}}
\newcommand{\awardedpaper}[2]{ {\centering \textbf{``#1''}} by {\centering #2.}}
\newcommand{\quot}[2]{{\vspace{2mm}``\emph{#1}''}}
\newcommand{\person}[4]{~\\~
\begin{minipage}{\textwidth}
\begin{wrapfigure}[10]{l}{2.15cm}
\vspace*{-1.2\baselineskip}%
\personphoto{#4}
\end{wrapfigure}
\emph{\small #3}
\end{minipage}
}

\ifodd\value{page}\hbox{}\newpage\fi


\section{SIGMOD Awards}

\subsection*{SIGMOD Test of Time Award}

\awardedpaper{Privacy integrated queries: an extensible platform for privacy-preserving data analysis}{Frank McSherry}

~\\~

The awards committee considers this paper a major scholastic contribution
in the way we should handle one of the core challenges in  data management. The paper
has become a landmark reference for research on a privacy preservation in these modern
times geared at automated data analysis with privacy implications.


\person{Frank McSherry}{Materialize, Inc.}{\textbf{Frank McSherry} is the Chief Scientist at Materialize, Inc., where he works on interactive and incremental data processing. Frank was previously at the Systems Group at ETH Z\"urich where he worked with students on timely and differential dataflow, and even further back at MSR Silicon Valley where he worked on the Naiad project and on Differential Privacy. Frank is perhaps best known for applying his undergraduate education to big data problems.}{images/awards/mcsherry.jpg}

\subsection*{SIGMOD Contribution Award}

The committee reached an unanimous decision to  grant the SIGMOD Contribution Award award to \textbf{Ahmed Elmagarmid}, especially for his dedicated service to the database community in North America and the Middle East, as the founder/editor of Distributed and Parallel Database Journal, and PC for ACM SIGMOD.


\person{Ahmed Elmagarmid}{Qatar Computing Research Institute, HBKU}{\textbf{Ahmed Elmagarmid} is the founding executive director of the Qatar Computing Research Institute and is an Emeritus Professor of Computer Science at Purdue University. He served as Chief Scientist at Hewlett Packard and Chief of Data Quality at Bellcore.  Dr. Elmagarmid is a recipient of the NSF Presidential Young Investigator (PYI) award from President Reagan in 1988. He is an IEEE Fellow, an ACM Fellow and an AAAS Fellow. The University of Dayton and Ohio State University have both named him among their distinguished alumni. His claim to fame is that he shared an office as graduate student with Prof. M.T.Ozsu.}{images/awards/elmagarmid.jpg}


\clearpage

\subsection*{SIGMOD E.F. Codd Innovation Award}

The awards committee selected \textbf{Anastasia Ailamaki} as the recipient of the 2019 ACM SIGMOD Edgar F. Codd Innovations Award for her pioneering work on the architecture of database systems, its interaction with computer architecture, and scientific data management.


\person{Anastasia Ailamaki}{Ecole Polytechnique Fédérale de Lausanne (EPFL) and RAW Labs SA}{\textbf{Anastasia Ailamaki} received her Ph.D. in Computer Science from the University of Wisconsin-Madison in 2000, was on the faculty of the School of Computer Science at Carnegie Mellon until 2008, and is currently a Professor of Computer and Communication Sciences at Ecole Polytechnique Fédérale de Lausanne (EPFL) in Switzerland.  In 2015, she co-founded RAW Labs SA, a swiss company that develops a real-time analytical engines that processes heterogeneous big data. Her research interests are in data-intensive systems and applications, and in particular (a) in strengthening the interaction between the database software and emerging hardware and I/O devices, and (b) in automating data management to support computationally-demanding, data-intensive scientific applications. She has received the Finmeccanica endowed chair from the Computer Science Department at Carnegie Mellon University (2007), an Alfred P. Sloan Research Fellowship (2005), and ten best-paper awards in database, storage, and computer architecture conferences. She has also received an ERC Consolidator Award (2013), a European Young Investigator Award from the European Science Foundation (2007), and an NSF CAREER award (2002). She is an ACM fellow, an IEEE fellow, the Laureate for the 2018 Nemitsas Prize in Computer Science, and an elected member of the Swiss, the Belgian, and the Cypriot National Research Councils.}{images/awards/ailamaki.jpg}



\clearpage

\subsection*{SIGMOD Jim Gray Doctoral Dissertation Award}

\textbf{Winner: Joy Arulraj}\vspace{1mm}
\person{Joy Arulraj}{Georgia Institute of Technology}{\textbf{Joy Arulraj} is an assistant Professor of Computer Science at Georgia Institute of Technology. He received his Ph.D. from Carnegie Mellon University in 2018, advised by Andy Pavlo. His doctoral research focused on the design and implementation of non-volatile memory database management systems. This work was conducted in collaboration with the Intel Science and Technology Center for Big Data, Microsoft Research, and Samsung Research.}{images/awards/arulraj.jpg}

\vspace{5mm}

\textbf{Honorable Mention: Bas Ketsman}
\vspace{1mm}
\person{Bas Ketsman}{Hasselt University}{\textbf{Bas Ketsman} is a postdoctoral researcher at the Swiss Federal Institute of Technology in Lausanne (EPFL). In 2017 he obtained his PhD from Hasselt University in Belgium, advised by Frank Neven, where he was a PhD fellow of the Research Foundations - Flanders (FWO) and a member of the Databases and Theoretical Computer Science group. His research focuses on foundational aspects of large-scale data management and distributed computing. Bas' papers were selected for the ACM SIGMOD Research Highlight Award, listed in ACM best of Computing, and appeared as research Highlight in CACM. He also received the Distinguished Dissertation Award 2018 from the European Association for Theoretical Computer Science (EATCS) and best paper awards at ACM PODS 2014 and 2015.}{images/awards/ketsman.jpg}

\clearpage

\subsection*{SIGMOD Best Paper Award}
\awardedpaper{Interventional Fairness : Causal Database Repair for Algorithmic Fairness}{Babak Salimi, Luke Rodriguez, Bill Howe, Dan Suciu}


\person{Babak Salimi}{University of Washington}{\textbf{Babak Salimi} is a postdoctoral research associate at University of Washington, where he works with Professor Dan Suciu. He received his Ph.D. from Carleton University, where he worked with Professor Leopoldo Bertossi. His research interests cover data management, decision making systems, causal inference and algorithmic fairness.}{images/awards/salimi.jpg}

~//~//~//~//~//

~//~//~//~//~//~

\person{Luke Rodriguez}{University of Washington}{\textbf{Luke Rodriguez} is a PhD Student advised by Bill Howe in the Information School at the University of Washington whose research focuses on using tools from differential privacy and causal reasoning to support scientific collaboration and responsible data management. In particular, Luke seeks to investigate how we can go beyond accessibility and availability and make data more useful.}{images/awards/rodriguez.jpg}

~//~//~//~//~

\person{Bill Howe}{University of Washington}{\textbf{Bill Howe} is Associate Professor in the Information School and Adjunct Associate Professor in the Allen School of Computer Science \& Engineering and the Department of Electrical Engineering. His research interests are in data management, curation, analytics, and visualization in the sciences. As Founding Associate Director of the UW eScience Institute, Howe played a leadership role in the Data Science Environment program at UW through a \$32.8 million grant awarded jointly to UW, NYU, and UC Berkeley, and founded UW's Data Science for Social Good Program. With support from the MacArthur Foundation and Microsoft, Howe directs UW's participation in the Cascadia Urban Analytics Cooperative, where he focuses on responsible data science. He founded the UW Data Science Masters Degree, serving as its inaugural Program Chair, and created a first MOOC on data science that attracted over 200,000 students. His research has been featured in the Economist and Nature News, and he co-authored what have remained the most-cited papers from VLDB 2010 and SIGMOD 2012. He has received two Jim Gray Seed Grant awards from Microsoft Research and two ``Best of Conference'' invited papers from VLDB Journal. He has a Ph.D. in Computer Science from Portland State University and a Bachelor's degree in Industrial \& Systems Engineering from Georgia Tech.}{images/awards/howe.png}


\person{Dan Suciu}{University of Washington}{\textbf{Dan Suciu} is a Professor in Computer Science at the University of Washington. He received his Ph.D. from the University of Pennsylvania in 1995, was a principal member of the technical staff at AT\&T Labs and joined the University of Washington in 2000. Suciu is conducting research in data management, with an emphasis on topics related to Big Data and data sharing, such as probabilistic data, data pricing, parallel data processing, data security. He is a co-author of two books Data on the Web: from Relations to Semistructured Data and XML, 1999, and Probabilistic Databases, 2011. He is a Fellow of the ACM, holds twelve US patents, received the best paper award in SIGMOD 2000 and ICDT 2013, the ACM PODS Alberto Mendelzon Test of Time Award in 2010 and in 2012, the 10 Year Most Influential Paper Award in ICDE 2013, the VLDB Ten Year Best Paper Award in 2014, and is a recipient of the NSF Career Award and of an Alfred P. Sloan Fellowship. Suciu serves on the VLDB Board of Trustees, and is an associate editor for the Journal of the ACM, VLDB Journal, ACM TWEB, and Information Systems and is a past associate editor for ACM TODS and ACM TOIS. Suciu's PhD students Gerome Miklau, Christopher Re and Paris Koutris received the ACM SIGMOD Best Dissertation Award in 2006, 2010, and 2016 respectively, and Nilesh Dalvi was a runner up in 2008.}{images/awards/suciu.jpg}


\clearpage


\section{PODS Awards}

\subsection*{PODS Albert O. Mendelzon Test of Time Award}


The ACM PODS Alberto O. Mendelzon Test-of-Time Award is awarded every year to a paper or a small number of papers published in the PODS proceedings ten years prior that had the most impact in terms of research, methodology, or transfer to practice over the intervening decade. After careful consideration and having solicited external nominations and advice, we have selected the following paper as the award winner for 2019:

\vspace{1mm}
\awardedpaper{A General Datalog-Based Framework for Tractable Query Answering over Ontologies}{Andrea Cali, Georg Gottlob, Thomas Lukasiewicz}
\vspace{1mm}

This paper introduces and studies the Datalog+/- framework for query answering over ontologies, which subsequently became highly influential in both the database and knowledge representation communities. Its main contribution is an in-depth study of the data complexity of Datalog+/-, and several extensions and restrictions tailored to ontologies. The paper identifies a tractable family of Datalog+/- formalisms, based on linear tuple-generating dependencies, that generalizes description logics of the DL-Lite family. Extensions with keys and stratified negation are also studied. Other technical results of the paper concerning the chase have been fundamental to further developments in the field. The paper received over 450 citations, evidencing its significant impact.

\person{Andrea Cali}{University of London, Birkbeck College}{\textbf{Andrea Cali} is a Senior Lecturer at the Department of Computer Science and Information Systems of the University of London, Birkbeck College. He holds a MEng in Electronic Engineering and a PhD in Computer Engineering, both from the University of Rome "La Sapienza". His research interests include Database Theory, Deep Web, Semantic Web and Ontology Reasoning, Information Integration and Linked Data querying. Among other interests, he investigates the computational complexity of fundamental problems in data processing under knowledge bases. His is currently researching how to automatically integrate Web data in decision support and matchmaking systems.}{images/awards/cali.jpg}



\person{Georg Gottlob}{University of Oxford and TU Wien}{\textbf{Georg Gottlob} is Professor of Informatics at Oxford and at TU Wien. His interests include database theory, AI, knowledge representation, logic and complexity, 
problem decompositions, and, on the more applied side, web data extraction, 
and practical database query processing. Gottlob has received the Wittgenstein Award 
(Austria)  and  the Ada Lovelace Medal (UK)., He   is an ACM Fellow, an ECCAI 
Fellow, a  Fellow of the Royal Society,  and a member of the Austrian and the German 
academies of  of Sciences, and the Academia Europaea.  He chaired the Program 
Committees of IJCAI 2003 and ACM PODS 2000. He was the main founder of Lixto, 
a web data extraction software company, which was acquired by McKinsey  
in 2013. Gottlob was awarded an ERC Advanced Investigator's Grant for the project 
"DIADEM: Domain-centric Intelligent Automated Data Extraction Methodology". Based 
on results of this project, he co-founded Wrapidity Ltd, a company that specialises 
in fully automated web data extraction that was recently acquired by the Meltwater
Media Intelligence corporation.}{images/awards/gottlob.jpg}

\person{Thomas Lukasiewicz}{University of Oxford}{\textbf{Thomas Lukasiewicz} is a Professor at the University of Oxford where he carries out research in the area of automatic web data extraction, such as the development of a general model of knowledge representation and reasoning for integrating different sources on the web. The work is directed towards possible key technologies for the future web, such as semantic searching. }{images/awards/lukasiewicz.jpg}




\clearpage

\subsection*{PODS Best Paper Award}

\awardedpaper{Efficient Logspace Classes for Enumeration, Counting, and Uniform Generation}{Marcelo Arenas,Luis Alberto Croqueviell, Rajesh Jayaram, Cristian Riveros}


\person{Marcelo Arenas}{Pontificia Universidad Catolica de Chile and Millennium Institute for Foundational Research on Data}{\textbf{Marcelo Arenas} is a Professor at the Department of Computer Science at the Pontificia Universidad Catolica de Chile, and the director of the Millennium Institute for Foundational Research on Data. He received a Ph.D. from the University of Toronto in 2005. His research interests are in the areas of data management, applications of logic in computer science and Semantic Web. Marcelo Arenas has received an IBM Ph.D. Fellowship (2004), a SIGMOD Jim Gray Doctoral Dissertation Award Honorable Mention in 2006 for his Ph.D. dissertation "Design Principles for XML Data", the 2016 Semantic Web Science Association (SWSA) Ten-Year Award for the article "Semantics and Complexity of SPARQL" and eight best paper awards (PODS 2003, PODS 2005, ISWC 2006, ICDT 2010, ESWC 2011, PODS 2011, WWW 2012 and ISWC 2014). He has served on multiple program committees and editorial boards, and he has chaired the program committees of ICDT 2015, ISWC 2015 and PODS 2018. }{images/awards/arenas.png}

\person{Luis Alberto Croqueviell}{Pontificia Universidad Catolica de Chile and Millennium Institute for Foundational Research on Data}{\textbf{Luis Alberto Croquevielle} is a student at Pontificia Universidad Católica de Chile, where he has just finished his Master of Science, advised by Marcelo Arenas. His master's research focused on the study of enumeration problems, and its relation to other questions such as counting and uniform generation. His work was supported by the Instituto Milenio de Investigación sobre los Fundamentos de los Datos (IMFD), and conducted in collaboration with other researchers of IMFD.}{images/awards/rendic.png}

\person{Rajesh Jayaram}{Pontificia Universidad Catolica de Chile and Millennium Institute for Foundational Research on Data}{\textbf{Rajesh Jayaram} is a PhD student in theoretical computer science at Carnegie Mellon University, advised by David Woodruff. Prior to beginning his PhD in 2017, Rajesh received his B.S. in Mathematics and Computer Science from Brown University. His research focuses primarily on streaming and sketching algorithms for problems in big-data, as well as lower bounds for these problems. In particular, his work frequently involves applying techniques from dimensionality reduction and the randomized analysis of algorithms to obtain efficient algorithms for big data-sets.
}{images/awards/jayaram.jpg}


\person{Rajesh Jayaram}{Pontificia Universidad Catolica de Chile and Millennium Institute for Foundational Research on Data}{\textbf{Cristian Riveros} is an assistant professor at the Department of Computer Science at PUC Chile and a young researcher at the Millenium Institute on Foundational Research on Data (IMFD). He received his D.Phil degree from the University of Oxford in 2013. Previously, he did his undergraduate studies at PUC Chile. His research interests are in database theory and data management systems, specifically, in data stream management systems, information extraction, and graph data. }{images/awards/riveros.jpg}

\clearpage

\subsection*{PODS Best Student Paper Award}
\awardedpaper{On the Enumeration Complexity of Unions of Conjunctive Queries}{Nofar Carmeli, Markus Kr\"{o}ll}

\person{Nofar Carmeli}{Technion, Israel Institute of Technology}{\textbf{Nofar Carmeli} is a PhD student in the Data and Knowledge group at Technion, Israel Institute of Technology, advised by Prof. Benny Kimelfeld.
Her research focuses on query optimization with guarantees using enumeration techniques.
Nofar completed her BSc in 2015 in the Lapidim excellence program of the Computer Science department of Technion.}{images/awards/carmeli.jpg}

\person{Markus Kr\"oll}{TU Wien}{\textbf{Markus Kr\"oll} is a pre-doctoral researcher at the TU Wien. In his research he focuses on database theory and lower bounds in enumeration complexity.}{images/awards/krol.jpg}



