%!TEX root = booklet.tex

\section{Welcome Message}

\subsection*{Welcome to ACM SIGMOD/PODS 2019!}

This year, SIGMOD/PODS is held in the city center of Amsterdam, capital of The Netherlands. Amsterdam is an internationally oriented city, home to people with origins from all over the world. This used to be already the case even back in the 16th and 17th century, when Amsterdam was the world's biggest trading and financial center; establishing the world's first stock exchange in 1602.

SIGMOD/PODS 2019 is held in the original Amsterdam Stock and Commodities Exchange, constructed by Dutch architect Berlage between 1896 and 1903, which now serves as the well-equipped Amsterdam Conference Center. This architect and his apprentices (the school of Berlage) left an important mark on the city, being responsible for a major expansion of the city in the early 20th century. The sculptures and drawings in the Exchange building refer to the people behind the commodities traded in the various rooms (``Effecten'' - stock; ``Graan'' - grains), e.g., depicting farmers in the grain exchange room; as a reminder that trading affects society.

Amsterdam is a city that offers many cultural activities, including the world-famous classical Concertgebouw Orchestra, as well as many museums (Amsterdam Museum, Rijksmuseum, Rembrandthuis, Anne Frank Huis). In a slight break with SIGMOD tradition, the SIGMOD opening reception will be held one day later, on Tuesday night, when the SIGMOD/PODS attendees will have exclusive access to the Van Gogh museum. The Wednesday conference dinner is organized across the water in Amsterdam North, in Noorderlicht Cafe in a festival-like environment. This used to be harbour area and was less-populated and industrial, but in the recent decade has become a hotspot for nightlife activities.

Amsterdam is also increasingly a hub for data science companies and services, with multiple universities and CWI in the vicinity; which all participate in the organization of SIGMOD/PODS 2019. On Thursday night, after the SIGMOD program finishes, there will be a meetup of Amsterdam Data Science, where the local data science community will be able to mingle with our data management research community.

\subsubsection*{Overview of SIGMOD 2019}

The SIGMOD 2019 Research Program Committee consists of the Program Chair, two Program Vice Chairs, a core committee with 37 members, and a regular committee with 98 members. During the reviewing period, we solicited additional reviews from 16 external reviewers and occasional input from 10 assistant reviewers. The committee received 430 submissions, out of which 12 were desk-rejected (i.e., without review). There was no bidding; instead, reviewer assignments were made using input from Microsoft's Conference Management System, the Toronto Paper Matching System, and the reviewers' background (the detailed assignment procedure is described in a paper which has been submitted for publication to SIGMOD Record). The core committee members had (roughly) double the reviewing load of the regular committee members, and in addition acted as discussion leaders and meta-reviewers for their assigned papers. There were two rounds of submissions, with deadlines in July and November, respectively. Initially, each paper received three reviews. At this point authors could read the reviews and provide feedback about potential factual errors (disclosed to the reviewers) or sensitive issues about potential mishandling (confidentially to the chair). Two additional reviews were solicited for a paper if (a) the reviewers' expertise level was suboptimal, or (b) if there was significant score discrepancy in the first three reviews, or (c) if it was heading for rejection but had received a weak accept (or higher) by at least one reviewer. Papers were discussed extensively online; 10 were accepted based on the first round of reviews, while 311 were rejected. The authors of the remaining 97 papers were asked to revise their papers to address reviewers' criticisms; 78 revisions were ultimately accepted for a total of 88 papers which are presented in the research track. Finally, 12 papers were shepherded after acceptance to guarantee that the camera-ready version addresses all of the reviewers' comments.

\subsubsection*{Overview of PODS 2019}

The PODS Program committee consists of 24 members, including the chair. PODS submissions received at least 4 reviews; papers that include PC members among their authors received at least 5 reviews, and higher standards apply for their acceptance.  As in previous years, Easychair was used as the conference management tool for PODS. Also, as in previous years, PODS operated with two submission cycles. The first cycle allowed for the possibility of papers being revised and resubmitted. For the first cycle, 36 papers were submitted, 4 of which were directly selected for inclusion in the proceedings, and 8 were invited for a resubmission after a revision. The quality of most of the revised papers increased substantially with respect to the first submission, and all of the revised papers were selected for the proceedings. For the second cycle, 51 papers were submitted, 17 of which were selected, resulting in 29 papers selected overall from a total number of 87 submissions. The Best Paper and Best Student Paper awards, as well as the Gems of PODS talks and invited tutorials, were selected by a subcommittee of the PC, while the Alberto-Mendelzon Test-of-Time award winners were chosen by a separate committee appointed by the PODS Executive Committee.

~\\~

{\small%
\setlength{\tabcolsep}{0pt}
\begin{tabular*}{\textwidth}{@{\extracolsep{\fill}}ll}
\textbf{Stefan Manegold, Peter Boncz} & \textbf{Dan Suciu}           \\
\emph{SIGMOD'19 General Chairs}       & \emph{PODS'19 General Chair} \\
CWI, Netherlands                      & University of Washington     \\
\\
\textbf{Anastasia Ailamaki}           & \textbf{Christoph Koch}      \\
\emph{SIGMOD'19 Program Chair}        & \emph{PODS'19 Program Chair} \\
EPFL, Switzerland                     & EPFL, Switzerland            \\
\end{tabular*}
}
